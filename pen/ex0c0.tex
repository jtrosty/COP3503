%  Make this into a pdf document as follows:
%
%
% The edit the Report.tex file appropriately to include only those elements that
% make sense for the assignment you're reporting on.
%
% You can use a tool like TeXShop or Texmaker or some other graphical tool
% to convert Report.text into a pdf fi le.
%
% If you are making this with command line tools, you'd run the following command:
%
%     latex Report.tex
%
% That will generate a dvi (device independent) document file called Report.dvi
% The pages reported in the table of contents won't be correct, since latex only
% processes one pass over the document. To adjust the page numbers in the contents,
% run latex again:
%
%    latex Report.tex
%
% Then run
%
%   dvipdf Report.dvi
%
% to generate Report.pdf
%
% You can view this file to check it out by running
%
% xdg-open Report.pdf
%
% That's it.

\def\cvss(#1,#2,#3,#4,#5,#6,#7,#8,#9){
	\indent\textbf{CVSS Base Severity Rating: #9}  AV:#1 AC:#2 PR:#3 UI:#4 S:#5 C:#6 I:#7 A:#8}

\def\ttp(#1, #2, #3, #4, #5, #6){
	\indent\textbf{#1:} #2 \\
	\indent\indent\textbf{#3:} #4 \\
	\indent\indent\indent\textbf{#5:} #6 \\}

\documentclass[notitlepage]{article}

\usepackage{bibunits}
\usepackage{comment}
\usepackage{graphicx}
\usepackage{amsmath}
\usepackage{datetime}
\usepackage{numprint}

% processes above options
\usepackage{palatino}  %OR newcent ncntrsbk helvet times palatino
\usepackage{url}
\usepackage{footmisc}
\usepackage{endnotes}
\usepackage{hyperref}

\urlstyle{same}

\setcounter{secnumdepth}{0}
\begin{document}
	
	\nplpadding{2}
	\newdateformat{isodate}{
		\THEYEAR -\numprint{\THEMONTH}-\numprint{\THEDAY}}
	
	\title{PenTest Lab Exercise Ex0c0 - Hiding in Plain Sight}
	\author{Jonathan Trost}
	\date{\isodate\today}
	
	\maketitle
	
	\tableofcontents
	
	\newpage 
	
	\section{Attack Narrative}
	\indent The goal of the attack in this report was to see if we could run penetration software on a windows machine within Arts Tailor. A connection to the computer at IP \textbf{books.artstailor.com}.  A remote desktop was opened on the at \textbf{books.artstailor.com} using the credentials that were gathered and reported in Ex090, the account used was \textbf{d.darkblood}. The service proxychains was setup to allow the a folder on the pr0b3 machine to the Art's Tailor machine.  \\
	\indent A \textbf{Veil} payload was attempted to get running on the Windows defender software detected the payload and prevented it from running. We were able to get \textbf{meterpeter} software running by initializing a server of it on the penetration testing computer. With \textbf{Meterpeter} up, the generated update file was transferred to \textbf{books.artstailor.com} this code was then run on the attacked computer.  Meterpeter was then fully running on the the penetration testing computer with access to the file system of books.artstailor.com. \\ 
	\includegraphics[width=5in]{meterpeter.jpg} \\
	
	 A key was found when inspecting the web code. \\
	
	\indent The code was \textbf{KEY013-qf/UZuKos+DzV6ot0Tktjg==} \\
	
	\subsection{MITRE ATT{\&}CK Framework TTPs}
	
	\indent\textbackslash ttp(TA0008, Lateral Movement, T1021, Remote Services, .001, Remote Desktop Protocol) \\
	\ttp(TA0008, Lateral Movement, T1021, Remote Services, .001, Remote Desktop Protocol) \\
	
	\indent\textbackslash ttp(TA0008, Reconnaissance, T1592, Gather Victim Host Information, .004, Client Configurations) \\
	\ttp(TA0003, Persistence, T1136, Create Account, .001, Local Account) \\


\end{document} 