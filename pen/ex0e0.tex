%  Make this into a pdf document as follows:
%
%
% The edit the Report.tex file appropriately to include only those elements that
% make sense for the assignment you're reporting on.
%
% You can use a tool like TeXShop or Texmaker or some other graphical tool
% to convert Report.text into a pdf fi le.
%
% If you are making this with command line tools, you'd run the following command:
%
%     latex Report.tex
%
% That will generate a dvi (device independent) document file called Report.dvi
% The pages reported in the table of contents won't be correct, since latex only
% processes one pass over the document. To adjust the page numbers in the contents,
% run latex again:
%
%    latex Report.tex
%
% Then run
%
%   dvipdf Report.dvi
%
% to generate Report.pdf
%
% You can view this file to check it out by running
%
% xdg-open Report.pdf
%
% That's it.

\def\cvss(#1,#2,#3,#4,#5,#6,#7,#8,#9){
	\indent\textbf{CVSS Base Severity Rating: #9}  AV:#1 AC:#2 PR:#3 UI:#4 S:#5 C:#6 I:#7 A:#8}

\def\ttp(#1, #2, #3, #4, #5, #6){
	\indent\textbf{#1:} #2 \\
	\indent\indent\textbf{#3:} #4 \\
	\indent\indent\indent\textbf{#5:} #6 \\}

\documentclass[notitlepage]{article}

\usepackage{bibunits}
\usepackage{comment}
\usepackage{graphicx}
\usepackage{amsmath}
\usepackage{datetime}
\usepackage{numprint}

% processes above options
\usepackage{palatino}  %OR newcent ncntrsbk helvet times palatino
\usepackage{url}
\usepackage{footmisc}
\usepackage{endnotes}
\usepackage{hyperref}

\urlstyle{same}

\setcounter{secnumdepth}{0}
\begin{document}
	
	\nplpadding{2}
	\newdateformat{isodate}{
		\THEYEAR -\numprint{\THEMONTH}-\numprint{\THEDAY}}
	
	\title{PenTest Lab Exercise Ex0e0 - SSLStrip}
	\author{Jonathan Trost}
	\date{\isodate\today}
	
	\maketitle
	
	\tableofcontents
	
	\newpage 
	
		\subsection{Summary of Findings}
	\indent No findings.
	
	\section{Attack Narrative}
	\indent I set up chisel, with the kali machine as server and costumes as proxy. I then used ssh and the credentials that were reported in report Ex0d0 to gain access to the linux devb\textbf{ox.artstailor.com.}  The user a.turing, has root access on the devbox.artstailor.com machine. I used \textbf{proxychains scp -r sslstrip-extras a.turing@artstailor.com:/~} to move over the sslstrip files to the \textbf{devbox.artstailor.com} machine. After the files were transferred I used \textbf{proxychains ssh a.turing@devbox.artstailor.com} to gain access to the devbox with the credentials previously gathered. From there the command \textbf{sudo -i} was used to be root on the machine.  I then began preparing the machine for SSL strip b y following the SSL strip README and setting up the ip\_forward to 1 while using the \textbf{sudo -i} command.  Then I setup the IP tables using the instructions from the SSLstrip README with the redirect port as 10000.  Modifications had to be made to SSLstrip in order to be run.  The probram was modified by change the line \textit{return b' '.join(headerComponent.splitlines())} to say r\textit{return b' '.join(str(headerComponent).splitlines())}.  
	\indent  With the SSLstrip mostly setup I used the \textbf{tcpdump} program that was transferred in the SSL folder to read packets of network information to \textbf{devbox.artstailor.com}.  The command used was \textbf{./tcpdump -i any -r packets.pcap}. The file was then transferred back to the kali machine to be read in wireshark witht eh command \textbf{proxychains scp -r a.turing@arstailor.com:/sslstrip/packets.pcap ~/}.  I then began to use wireshark to determine which host to perform the SSLstrip man in the middle attack on but ran out of time before this was determined. 
	

	
	\subsection{MITRE ATT{\&}CK Framework TTPs}
	
	\indent\textbackslash ttp(TA0006, Credential Access, T1552, Unsecured Credentials, .001, Credentials In Files) \\
	\ttp(A0006, Credential Access, T1552, Unsecured Credentials, .001, Credentials In Files) \\
	
	\indent\textbackslash ttp(TA0007, Discovery, T1040, Network Sniffing, TCPDUMP, TCUPDUMP) \\
	\ttp(TA0007, Discovery, T1040, Network Sniffing, TCPDUMP, TCUPDUMP) \\


\end{document} 