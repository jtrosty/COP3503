%  Make this into a pdf document as follows:
%
%
% The edit the Report.tex file appropriately to include only those elements that
% make sense for the assignment you're reporting on.
%
% You can use a tool like TeXShop or Texmaker or some other graphical tool
% to convert Report.text into a pdf file.
%
% If you are making this with command line tools, you'd run the following command:
%
%     latex Report.tex
%
% That will generate a dvi (device independent) document file called Report.dvi
% The pages reported in the table of contents won't be correct, since latex only
% processes one pass over the document. To adjust the page numbers in the contents,
% run latex again:
%
%    latex Report.tex
%
% Then run
%
%   dvipdf Report.dvi
%
% to generate Report.pdf
%
% You can view this file to check it out by running
%
% xdg-open Report.pdf
%
% That's it.

\def\cvss(#1,#2,#3,#4,#5,#6,#7,#8,#9){
	\indent\textbf{CVSS Base Severity Rating: #9}  AV:#1 AC:#2 PR:#3 UI:#4 S:#5 C:#6 I:#7 A:#8}

\def\ttp(#1, #2, #3, #4, #5, #6){
	\indent\textbf{#1:} #2 \\
	\indent\indent\textbf{#3:} #4 \\
	\indent\indent\indent\textbf{#5:} #6 \\}

\documentclass[notitlepage]{article}

\usepackage{bibunits}
\usepackage{comment}
\usepackage{graphicx}
\usepackage{amsmath}
\usepackage{datetime}
\usepackage{numprint}

% processes above options
\usepackage{palatino}  %OR newcent ncntrsbk helvet times palatino
\usepackage{url}
\usepackage{footmisc}
\usepackage{endnotes}

\setcounter{secnumdepth}{0}
\begin{document}
	
	\nplpadding{2}
	\newdateformat{isodate}{
		\THEYEAR -\numprint{\THEMONTH}-\numprint{\THEDAY}}
	
	\title{PenTest Lab Exercise Ex010 - Kali Netlab}
	\author{Jonathan Trost}
	\date{\isodate\today}
	
	\maketitle
	
	\tableofcontents
	
	\newpage
	\section{Attack Narrative}
	
	\subsection{KEY001}
	The first key was described as being part of a file name. To search all files for the KEY I constructed the below \textbf{grep} and executed it in the user's home directory. I used regular expressions to search for strings in the format specified for the keys. \\
	\indent grep -rE "(KEY[0-9]{3}-[a-zA-Z0-9+/]{22})" \\
	The result is KEY001-R2f6GgmJusEGksUUhIsyHQ== \\
 
	\subsection{KEY002}
	The bold \textbf{process} in the instructions hinted that the key would be listed among processes. So I used the following command and search looked through the output. The \textbf{ps} command provides a list of current process, the \textbf{a} and \textbf{x} flags together have the ps command list all processes. The \textbf{x} flag list all process own by you and the \textbf{a} lists all processes with a terminal according to the man pages. \\
	\indent ps ax \\
	The result is KEY002-5bnFi7p7u8VUky1q6h0xyA==
\end{document} 