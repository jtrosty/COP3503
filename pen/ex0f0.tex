%  Make this into a pdf document as follows:
%
%
% The edit the Report.tex file appropriately to include only those elements that
% make sense for the assignment you're reporting on.
%
% You can use a tool like TeXShop or Texmaker or some other graphical tool
% to convert Report.text into a pdf fi le.
%
% If you are making this with command line tools, you'd run the following command:
%
%     latex Report.tex
%
% That will generate a dvi (device independent) document file called Report.dvi
% The pages reported in the table of contents won't be correct, since latex only
% processes one pass over the document. To adjust the page numbers in the contents,
% run latex again:
%
%    latex Report.tex
%
% Then run
%
%   dvipdf Report.dvi
%
% to generate Report.pdf
%
% You can view this file to check it out by running
%
% xdg-open Report.pdf
%
% That's it.

\def\cvss(#1,#2,#3,#4,#5,#6,#7,#8,#9){
	\indent\textbf{CVSS Base Severity Rating: #9}  AV:#1 AC:#2 PR:#3 UI:#4 S:#5 C:#6 I:#7 A:#8}

\def\ttp(#1, #2, #3, #4, #5, #6){
	\indent\textbf{#1:} #2 \\
	\indent\indent\textbf{#3:} #4 \\
	\indent\indent\indent\textbf{#5:} #6 \\}

\documentclass[notitlepage]{article}

\usepackage{bibunits}
\usepackage{comment}
\usepackage{graphicx}
\usepackage{amsmath}
\usepackage{datetime}
\usepackage{numprint}

% processes above options
\usepackage{palatino}  %OR newcent ncntrsbk helvet times palatino
\usepackage{url}
\usepackage{footmisc}
\usepackage{endnotes}
\usepackage{hyperref}

\urlstyle{same}

\setcounter{secnumdepth}{0}
\begin{document}
	
	\nplpadding{2}
	\newdateformat{isodate}{
		\THEYEAR -\numprint{\THEMONTH}-\numprint{\THEDAY}}
	
	\title{PenTest Lab Exercise Ex0f0 - Linux is Broken}
	\author{Jonathan Trost}
	\date{\isodate\today}
	
	\maketitle
	
	\tableofcontents
	
	\newpage 
	
	\subsection{Summary of Findings}
	\indent Pr0b3 was able to get root access on devbox.artstailor.com using the ability of a.turing to open \textbf{ps} with root and modifying paths to the \textbf{ps} file. Pr0b3 discovered that o\textbf{opp} folder contains a version of \textbf{samba} that is vulnerable to \textit{Eternal Red} see \textbf{CVE-2017-7494}.
	
	\subsection{Finding: \emph{Escalation of Privileges with Sudo}}
	
	\subsubsection{Severity Rating}
	\indent The severity is 8.8 This is a high risk exploit that opens network access to an attacker. 
	
	\textbackslash cvss(N,L, L, N, U, H, H, H, 8.8)\\
	\cvss(N,L, L, N, U, H, H, H, 8.8) \\
	
	\subsubsection{Vulnerability Description}
	\indent If an attacker gains access to an employee account that does not have root privileges they can potentially gain root privileges.  Pr0b3 inspected the \textbf{sudodoers} file and observed that a.turing has the ability to run /usr/bin/ps with root privileges.  When a.turing calls for commands in command shell, the first file location that is checked is \textbf{/home/a.turing/bin}.  Pr0b3 copied \textbf{bash} into that folder and remained it psreal.  Upon executing th
	
	\subsubsection{Confirmation method}
	To confirm the exploit use: \\
	\indent 1. Copy bash into a.turing's /bin folder. cp /usr/bin/bash /home/a.turing/bin.
	\indent 2. Rename the file to psreal. \textbf{cp bash psreal}.
	\indent 3. Run command \textbf{sudo -u\#-1 /usr/bin/ps} and enter a.turing's password.  
	\indent 3. You now have a root on devbox.artstailors.com.
	
	\subsubsection{Mitigation or Resolution Strategy}
	\indent \indent 1. Removes a.turings's ability to run \textbf{usr/bin/ps} as root. 
	
	\subsection{Finding: \emph{Eternal Red}}
	
	\subsubsection{Severity Rating}
	\indent The severity is 9.8 This is a high risk exploit the allows attackers to remote run code on the attacked machine. 
	
	\textbackslash cvss(N,L, N, N, U, H, H, H, 9.8)\\
	\cvss(N,L, N, N, U, H, H, H, 9.8) \\
	
	\subsubsection{Vulnerability Description}
	\indent Samba since version 3.5.0 and before 4.6.4, 4.5.10 and 4.4.14 is vulnerable to remote code execution vulnerability, allowing a malicious client to upload a shared library to a writable share, and then cause the server to load and execute it.
	
	\subsubsection{Confirmation method}
	To confirm the exploit use: \\
	\indent 1. Observe that the samba version is 4.5.9. Versions before 4.5.10 are vulnerable. 
	
	\subsubsection{Mitigation or Resolution Strategy}
	\indent Update to the latest Samba Revision.  
	
	\section{Attack Narrative}
	\indent Using the previously acquired credentials of a.turing, access was gained to devbox.artstailor.com.  Pr0b3 was able to escalate privileges to \textbf{root} by observing \textbf{a.turing} has the ability to run\textbf{ /usr/bin/ps} as root.  When a.turing calls for commands in command shell, the first file location that is checked is \textbf{/home/a.turing/bin}.  Pr0b3 copied \textbf{bash} into that folder and remained it psreal.  With command \textbf{sudo -u\#-1 /usr/bin/ps} a root terminal was opened. With the privileges, pr0b3 went through the file system and found that the version of samba Oppenhiemer has is vulnerable to the \textbf{Eternal Red} vulnerability \textbf{ CVE-2017-7479.} \\
	
	\indent In the below screen grab, one can see that the command \textbf{psreal} does not have a defined path.  This means that we can put a file named \textbf{psreal} in one of the directories that bash looks through to find executables. Also, youc an see that pr0b3 was able to make the changes and gain root access on devbox.   
	
	\includegraphics[width=4in]{f_0.png}
	
	\indent At the top of the files system, in the root folder a files, MyDream.png was found that contained a key. \\
	\textbf{KEY017-NiJFMBPA+1NQugW4CtcBdw==} \\
	
	\indent In the below picture you can see the samba files in the opp directory and their versions.  
	
	
	
	\includegraphics[width=4in]{f_1.png}
	
	\subsection{MITRE ATT{\&}CK Framework TTPs}
	
	\indent\textbackslash ttp(TA0004, Privilege Escalation, T1574, Hijack Execution Flow, .007, Path Interception by PATH Environment Variable) \\
	\ttp(TA0004, Privilege Escalation, T1574, Hijack Execuation Flow, .007, Path Interception by PATH Environment Variable) \\
	


\end{document} 