%  Make this into a pdf document as follows:
%
%
% The edit the Report.tex file appropriately to include only those elements that
% make sense for the assignment you're reporting on.
%
% You can use a tool like TeXShop or Texmaker or some other graphical tool
% to convert Report.text into a pdf fi le.
%
% If you are making this with command line tools, you'd run the following command:
%
%     latex Report.tex
%
% That will generate a dvi (device independent) document file called Report.dvi
% The pages reported in the table of contents won't be correct, since latex only
% processes one pass over the document. To adjust the page numbers in the contents,
% run latex again:
%
%    latex Report.tex
%
% Then run
%
%   dvipdf Report.dvi
%
% to generate Report.pdf
%
% You can view this file to check it out by running
%
% xdg-open Report.pdf
%
% That's it.

\def\cvss(#1,#2,#3,#4,#5,#6,#7,#8,#9){
	\indent\textbf{CVSS Base Severity Rating: #9}  AV:#1 AC:#2 PR:#3 UI:#4 S:#5 C:#6 I:#7 A:#8}

\def\ttp(#1, #2, #3, #4, #5, #6){
	\indent\textbf{#1:} #2 \\
	\indent\indent\textbf{#3:} #4 \\
	\indent\indent\indent\textbf{#5:} #6 \\}

\documentclass[notitlepage]{article}

\usepackage{bibunits}
\usepackage{comment}
\usepackage{graphicx}
\usepackage{amsmath}
\usepackage{datetime}
\usepackage{numprint}

% processes above options
\usepackage{palatino}  %OR newcent ncntrsbk helvet times palatino
\usepackage{url}
\usepackage{footmisc}
\usepackage{endnotes}

\setcounter{secnumdepth}{0}
\begin{document}
	
	\nplpadding{2}
	\newdateformat{isodate}{
		\THEYEAR -\numprint{\THEMONTH}-\numprint{\THEDAY}}
	
	\title{PenTest Lab Exercise Ex060 - VulnScan}
	\author{Jonathan Trost}
	\date{\isodate\today}
	
	\maketitle
	
	\tableofcontents
	
	\newpage
	
	\subsection{Summary of Findings}
	\indent Pr0b3 performed reconnaissance on Art's Tailor shops network to find open ports and services running. A vulnerability scan was performed on the active services. One of the running services was found to have a  serious exploit that requires corrective action.  
	
	\subsection{Finding: \emph{Backdoor on File Transfer Server, vsftpd 2.3.4}}
	
	\subsubsection{Severity Rating}
	The severity is 9.8. This is a high risk exploit that opens a shell on the root on the machine.  This exploit would allow the attacker to take data, delete data, and bring down the system or install additional tools.  

	\textbackslash cvss(N,L,N,N,U,H,H,H,9.8)\\
	\cvss(N,L,N,N,U,H,H,H,9.8) \\
	
	\subsubsection{Vulnerability Description}
	The vulnerability opens a shell on port 6200/tcp. This allows the user to perform any actions on the system a user could. The attacker could access secure files, delete data, or bring down the system. This vulnerability has a high impact for confidentiality, integrity, and availability of th system and data. 
	
	\subsubsection{Confirmation method}
	To confirm the exploit use: \\
	\indent 1. Open metasploit and search for vsftp. There should be one exploit listed. Set the exploit with the \textbf{use} keyword.\\
	\indent 2. Set the RHOST IP address of 217.70.184.38. \\
	\indent 3. Execute the exploit with the keyword \textbf{exploit}.\\
	
	
	\subsubsection{Mitigation or Resolution Strategy}
	Download and use the latest version of vsftpd. The back door and been removed. \\
	
	\section{Attack Narrative}
	A scan was performed with the Nessus system which found two vulnerabilities that are classified as high risk.  \\
	1. vsftpd Smiley Face Backdoor, has a exploit available. \\
	2. OpenSSH 8.2 < 8.5, no exploit available. \\
	3. DNS Server Spoofed Request Amplification DDoS. no exploit available. \\
	The "vsftpd Smiley Face Backdoor" was used to gain access to the system with IP address 217.70.184.38. A search of the system found the key
	\textbf{KEY008-HHAw+K7/1A+SR/Edya9kEw==}. The exploit used the username \textbf{l2:)} with password \textbf{uWP}. There is a programmed back door that when a username with a smiley face "\textbf{:)}" is used to gain access through the system.  The exploit then opens a shell through a unused port.  
	
	\subsection{MITRE ATT{\&}CK Framework TTPs}
	
	\indent\textbackslash ttp(T0043, Reconnaissance, T1595, Active Scanning, .002, Vulnerability Scanning) \\
	
	\ttp(TA0043, Reconnaissance, T1595, Active Scanning, .002, Vulnerability Scanning) \\

\end{document} 