%  Make this into a pdf document as follows:
%
%
% The edit the Report.tex file appropriately to include only those elements that
% make sense for the assignment you're reporting on.
%
% You can use a tool like TeXShop or Texmaker or some other graphical tool
% to convert Report.text into a pdf fi le.
%
% If you are making this with command line tools, you'd run the following command:
%
%     latex Report.tex
%
% That will generate a dvi (device independent) document file called Report.dvi
% The pages reported in the table of contents won't be correct, since latex only
% processes one pass over the document. To adjust the page numbers in the contents,
% run latex again:
%
%    latex Report.tex
%
% Then run
%
%   dvipdf Report.dvi
%
% to generate Report.pdf
%
% You can view this file to check it out by running
%
% xdg-open Report.pdf
%
% That's it.

\def\cvss(#1,#2,#3,#4,#5,#6,#7,#8,#9){
	\indent\textbf{CVSS Base Severity Rating: #9}  AV:#1 AC:#2 PR:#3 UI:#4 S:#5 C:#6 I:#7 A:#8}

\def\ttp(#1, #2, #3, #4, #5, #6){
	\indent\textbf{#1:} #2 \\
	\indent\indent\textbf{#3:} #4 \\
	\indent\indent\indent\textbf{#5:} #6 \\}

\documentclass[notitlepage]{article}

\usepackage{bibunits}
\usepackage{comment}
\usepackage{graphicx}
\usepackage{amsmath}
\usepackage{datetime}
\usepackage{numprint}

% processes above options
\usepackage{palatino}  %OR newcent ncntrsbk helvet times palatino
\usepackage{url}
\usepackage{footmisc}
\usepackage{endnotes}
\usepackage{hyperref}
\hypersetup{
	colorlinks=true,
	linkcolor=blue,
	filecolor=magenta,      
	urlcolor=cyan,
	pdftitle={Overleaf Example},
	pdfpagemode=FullScreen,
}
\urlstyle{same}

\setcounter{secnumdepth}{0}
\begin{document}
	
	\nplpadding{2}
	\newdateformat{isodate}{
		\THEYEAR -\numprint{\THEMONTH}-\numprint{\THEDAY}}
	
	\title{PenTest Lab Exercise Ex0b0 - Pivot}
	\author{Jonathan Trost}
	\date{\isodate\today}
	
	\maketitle
	
	\tableofcontents
	
	\newpage \\
	
	\section{Attack Narrative}
	\indent Accessed COSTUMES computer with provided admin credentials. Then I setup chisel server on kali attack machine. The client was then setup on the costumes machine. The chisel server was set up with --reverse on port 8000 with socks5.  The chisel client then was setup to access the server with R:socks. After having the chisel server and client set up, proxychains was used to run nmap on the devbox.artstailor.com to check port 80. A server was found running. The server was opened in firefox on the kali machine using \textbf{localhost}. The following graphic is proof we were able to access the webpage run on devbox.artstailor.com. \\
	\includegraphics[width=4in]{ex0b0.jpg} \\
	
	 A key was found when inspecting the web code. \\
	
	\indent The code was \textbf{KEY012-nrmhB1ncN1rMu0SZrpuMQg==} \\
	
	\subsection{MITRE ATT{\&}CK Framework TTPs}
	
	\indent\textbackslash ttp(TA0008, Lateral Movement, T1021, Remote Services, .001, Remote Desktop Protocol) \\
	\ttp(TA0008, Lateral Movement, T1021, Remote Services, .001, Remote Desktop Protocol) \\
	
	\indent\textbackslash ttp(TA0008, Reconnaissance, T1592, Gather Victim Host Information, .004, Client Configurations) \\
	\ttp(TA0003, Persistence, T1136, Create Account, .001, Local Account) \\


\end{document} 