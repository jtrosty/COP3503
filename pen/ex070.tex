%  Make this into a pdf document as follows:
%
%
% The edit the Report.tex file appropriately to include only those elements that
% make sense for the assignment you're reporting on.
%
% You can use a tool like TeXShop or Texmaker or some other graphical tool
% to convert Report.text into a pdf fi le.
%
% If you are making this with command line tools, you'd run the following command:
%
%     latex Report.tex
%
% That will generate a dvi (device independent) document file called Report.dvi
% The pages reported in the table of contents won't be correct, since latex only
% processes one pass over the document. To adjust the page numbers in the contents,
% run latex again:
%
%    latex Report.tex
%
% Then run
%
%   dvipdf Report.dvi
%
% to generate Report.pdf
%
% You can view this file to check it out by running
%
% xdg-open Report.pdf
%
% That's it.

\def\cvss(#1,#2,#3,#4,#5,#6,#7,#8,#9){
	\indent\textbf{CVSS Base Severity Rating: #9}  AV:#1 AC:#2 PR:#3 UI:#4 S:#5 C:#6 I:#7 A:#8}

\def\ttp(#1, #2, #3, #4, #5, #6){
	\indent\textbf{#1:} #2 \\
	\indent\indent\textbf{#3:} #4 \\
	\indent\indent\indent\textbf{#5:} #6 \\}

\documentclass[notitlepage]{article}

\usepackage{bibunits}
\usepackage{comment}
\usepackage{graphicx}
\usepackage{amsmath}
\usepackage{datetime}
\usepackage{numprint}

% processes above options
\usepackage{palatino}  %OR newcent ncntrsbk helvet times palatino
\usepackage{url}
\usepackage{footmisc}
\usepackage{endnotes}

\setcounter{secnumdepth}{0}
\begin{document}
	
	\nplpadding{2}
	\newdateformat{isodate}{
		\THEYEAR -\numprint{\THEMONTH}-\numprint{\THEDAY}}
	
	\title{PenTest Lab Exercise Ex070 - Brian's Service}
	\author{Jonathan Trost}
	\date{\isodate\today}
	
	\maketitle
	
	\tableofcontents
	
	\newpage
	
	\subsection{Summary of Findings}
	\indent Pr0b3 performed exploit investigation of customer service made by the customer.  A memory overflow bug was found and exploited that allows attacker to open a shell.
	
	\subsection{Finding: \emph{Buffer Overflow Exploit on Custom Service 'Toool'}}
	
	\subsubsection{Severity Rating}
	\indent The severity is 9.8. This is a high risk exploit that opens a shell on the root on the machine.  This exploit would allow the attacker to take data, delete data, and bring down the system or install additional tools.  

	\textbackslash cvss(N,L,N,N,U,H,H,H,9.8)\\
	\cvss(N,L,N,N,U,H,H,H,9.8) \\
	
	\subsubsection{Vulnerability Description}
	\indent The vulnerability opens a shell on port 6200/tcp. This allows the user to perform any actions on the system a user could. The attacker could access secure files, delete data, or bring down the system. This vulnerability has a high impact for confidentiality, integrity, and availability of th system and data. 
	
	\subsubsection{Confirmation method}
	To confirm the exploit use: \\
	\indent 1. Use nmap to inspect PORT 1337 using with flags for TCP.\\
	\indent 2. Use netcat to establish chat with computer, command is \textbf{nc 217.70.184.38 1337}.  \\
	\indent 3. With connection open a prompt will come up, enter the username \textbf{brian} and press enter.\\
	\indent 4. 3 options should be present, \textbf{ip a}, \textbf{ps auxww}, and \textbf{netstat -nat}. \\
	\indent 5. Now perform the buffer over flow.  Type 29 characters, then type \textbf{\textbackslash bin\textbackslash sh} and press enter. Then type \textbf{\textbackslash bin\textbackslash sh} again. The shell is now open.  \\
	
	
	\subsubsection{Mitigation or Resolution Strategy}
	\indent Remove service from use. If intending to continue to use service, ensure the second arguments of \textbf{fgets} is small enough to not go beyond the bounds of the 1st arguments memory. \\
	
	\section{Attack Narrative}
	\indent A scan was performed with the nmap with a expanded ports list in order to detect port 1337.  A connection was then made to port 1337 with netcat. From there a buffer overflow attack was used to override the options with ea command to open a shell. The buffer overflow attack occurs because of the \textbf{BUFLEN} given to the \textbf{fgets} function is 1024, so a user can input up to 1024 characters in the \textbf{next\_command} array.  \\ 
	\indent The commands array holds 37 char bytes. There are 12 byte increments that get a command in the \textbf{command} array. The command array is in close proximity to the \textbf{next\_command} array that takes in the user input. Assuming no padding, the \textbf{command} array is 29 bytes from the start of \textbf{next\_command}, 13 bytes for \textbf{next\_command} and 16 bytes for \textbf{admin}. If user input is greater than 29 bytes, the \textbf{command} array will be overridden. This allowed our team to insert a shell command into this place. \\
  
	
	\subsection{MITRE ATT{\&}CK Framework TTPs}
	
	\indent\textbackslash ttp(TA0002, Execution, T1059, Command and Scripting Interpreter, .008, Network Device CLI) \\
	
	\ttp(TA0002, Execution, T1059, Command and Scripting Interpreter, .008, Network Device CLI) \\

\end{document} 